\textbackslash documentclass\{article\} \textbackslash usepackage\{CJK\}
\textbackslash usepackage\{fontspec\}
\textbackslash usepackage\{graphicx\} \textbackslash setmainfont\{Noto
Serif CJK SC\} \textbackslash usepackage{[}top=2cm, bottom=2cm,
left=3cm, right=3cm, a4paper{]}\{geometry\}

\textbackslash begin\{document\}

enable exchange of product information between PLM applications (for
example, between a CAD application and a CAE application). They also
enable exchange of product information between PLM applications and
other enterprise applications such as ERP and CRM. In a very relevant
fashion, this middleware line of thinking is expanded upon by (Ben
Khedher et al., 2011). In their work regarding different systems
architectures for the implementation of an integrated MES+PLM they
describe the use of a mediation system in web service architecture. As
depicted in Figure 8, the proposed architecture uses data exchange based
on internet technologies to help companies, especially expanded
companies, to take advantage of opportunities generated by the Web
Services. The concept of "web service" means an application (program or
software system) which is designed to support interoperable
machine-to-machine interactions over a network, according to the
definition of W3C (Ben Khedher et al., 2011). The reason this expansion
is so relevant from the perspective of this work is that the Odoo
software works in a similar fashion through a similar web service
architecture. In theory the Odoo software could act as the middleware
working through the local network or hosted in the cloud and enacting
the layer of integration that was previously mentioned.

啟用 PLM 應用程式之間的產品資訊交換(例如,CAD 應用程式和 CAE
應用程式之間)。它們\textbackslash\textbackslash 還啟用 PLM
應用程式與其他企業應用程式(如 ERP 和 CRM)之間的產品資訊交換。

在非常相關的方面,(Ben Khedher 等人,2011
年)進一步擴展了這種中介軟體的思路。在\textbackslash\textbackslash 他們關於實施集成
MES+PLM 的不同系統架構的工作中,他們描述了在 Web
服務架構中使用中介\textbackslash\textbackslash 系統。正如圖 8
所示,所提出的架構使用基於互聯網技術的數據交換來幫助公司,特別是擴展公司,\textbackslash\textbackslash 利用
Web 服務產生的機會。``Web
服務''這一概念指的是一種應用程式(程序或軟體系統),根\textbackslash\textbackslash 據
W3C 的定義,它設計用來支持在網絡上進行可互操作的機器對機器的交互(Ben
Khedher 等\textbackslash\textbackslash 人,2011 年)。

從這項工作的角度來看,這種擴展之所以如此相關,是因為 Odoo 軟體通過類似的
Web 服務架\textbackslash\textbackslash 構以類似的方式運作。理論上,Odoo
軟體可以充當中介軟體,通過本地網絡或雲端托管來運作,\textbackslash\textbackslash 實現前面提到的集成層。

\textbackslash begin\{center\} \textbackslash centering
\textbackslash includegraphics{[}width=1\textbackslash linewidth{]}\{image.png\}
\textbackslash end\{center\}

3.1. How would this integration look like in practical terms

As mentioned in CHAPTER 2 the main idea of PLM is to manage change in
all processes related to the product, and it does so mainly through the
use of virtualization. The word virtualization here denotes
representation of item of the real world to the digital space and, as
one can imagine, there are several levels of abstraction through which a
real object or process can be represented. As consequence there is no
exact consensus regarding PLM of how deep and/or detailed the virtual
representation must be to serve its purpose. In an ideal world that
would be the lowest form of abstraction which, essentially, would come
down to a digital twin as explained in the CHAPTER 2. This is a `1 to 1'
digital representation of every aspect of the production cycle where
every part involved would have a digital representation that not only
carry the physical characteristics of the item but also all its
information produced over time. To this end, as explained in CHAPTER 2,
MES takes a fundamental role in obtaining the real time information
required for the DT even be possible. For instance, a CNC machine would
have a digital 3D model for simulation as well as a fully integrated
list of all the pieces it produces, data regarding its current level of
production, the current wear of its mechanical pieces, all other
machines it relates to, history of all the alterations and improvements
by which it was affected and many other aspects, all well packaged in an
intuitive graphical user interface (GUI) that allows for maximum
interaction. Outside of fiction, we are yet to achieve such level of
virtualization. It takes too much time and money to obtain and organize
information to such a level of minutia, specially, the aspects that need
to be inserted by hand, not to mention the subjectiveness of how this
information can be integrated and interacted with. Regardless of that it
is useful to identify, within the ideal, the aspects of most importance
for this implementation. \textbackslash\textbackslash Those are:
\textbackslash\textbackslash▪ The means of virtualization -- What sort
of information is used to build the virtual items. This includes the
metadata and files that are directly attached to the item. In an ideal
fashion this would contain all possible information available about the
item. \textbackslash\textbackslash▪ The means of data input - How this
information is being loaded and organized. Ideally this information
would be loaded into the system as automatically as possible, be it by
means of MES during quality control or through the use of automated
input tools like bar code scanners. \textbackslash\textbackslash▪ The
means of access -- How this information is presented to the users.
Although more subjective than the previous aspects this is incredibly
important to the way the system is interacted with. How intuitive it is
the information availability plays right into the core strengths of PLM.
Afterall, everything would be for nothing (even if all else would be
perfect) if the only way to interact with the system were a command line
interface that would make difficult for the end users to access the
information. \textbackslash\textbackslash▪ The means of integration -
How items and their contained information can interact and benefit from
one another, i.e., the integration with other systems and key softwares.
E.g., if an item has access to a cad file, there should be no need to
fill in the metadata fields by hand. Hoe items can automatically affect
other items also plays into this aspect.
\textbackslash\textbackslash3.1.這種整合在實際中會是什麼樣\textbackslash\textbackslash{}
如在第二章中提到的,PLM
的主要理念是管理與產品相關的所有過程中的變更,這主要通過虛\textbackslash\textbackslash 擬化來實現。這裡的虛擬化一詞指的是將現實世界中的項目表示為數字空間中的對象,正如可以想\textbackslash\textbackslash 像的那樣,實際對象或過程可以通過幾個抽象層次來表示。因此,關於
PLM,沒有關於虛擬表示\textbackslash\textbackslash 必須有多深入和/或詳細才能達到其目的的確切共識。

在理想的情況下,這將是最低形式的抽象,基本上會歸結為第二章中解釋的數字孿生體。這是\textbackslash\textbackslash 一種生產週期每個方面的``1
比
1''的數字表示,其中每個涉及的部分都會有一個數字表示,不僅\textbackslash\textbackslash 承載該項目的物理特徵,還包括其隨時間產生的所有信息。為此,如第二章所述,MES
在獲取數\textbackslash\textbackslash 字孿生體所需的實時信息方面起著基本作用。

例如,一台 CNC 機器將有一個用於模擬的數字 3D
模型,以及一個完整集成的其生產的所有\textbackslash\textbackslash 零件列表,關於其當前生產水平的數據,機械零件的當前磨損情況,它相關的所有其他機器,其\textbackslash\textbackslash 經歷的所有更改和改進的歷史以及許多其他方面,所有這些都打包在一個直觀的圖形用戶界面\textbackslash\textbackslash(GUI)中,允許最大限度的互動。

在現實世界中,我們尚未達到這種虛擬化的水平。獲取和組織如此細緻的資訊需要太多的時間\textbackslash\textbackslash 和金錢,特別是需要手動插入的方面,更不用說這些資訊如何集成和互動的主觀性。不管怎樣,\textbackslash\textbackslash 識別在理想中對實施最重要的方面是有用的。

這些方面是: \textbackslash\textbackslash▪虛擬化的手段 --
用於構建虛擬項目的資訊種類。這包括直接附加到項目的元數據和文件。在理\textbackslash\textbackslash 想情況下,這將包含關於該項目的所有可能的資訊。
\textbackslash\textbackslash▪資料輸入的手段 --
如何加載和組織這些資訊。理想情況下,這些資訊將盡可能自動地加載到系統\textbackslash\textbackslash 中,無論是通過
MES 在質量控制期間,還是通過使用條碼掃描儀等自動輸入工具。
\textbackslash\textbackslash▪訪問的手段 --
如何向用戶呈現這些資訊。儘管比前面的方面更主觀,但這對於系統的互動方式至\textbackslash\textbackslash 關重要。資訊的可用性有多直觀,直接關係到
PLM
的核心優勢。畢竟,即使所有其他方面都很完\textbackslash\textbackslash 美,如果唯一的交互方式是一個使最終用戶難以訪問資訊的命令行界面,那麼一切都將無濟於事。
\textbackslash\textbackslash▪集成的手段 --
項目及其所包含的資訊如何互動並相互受益,即與其他系統和關鍵軟體的集成。例\textbackslash\textbackslash 如,如果一個項目可以訪問
CAD
文件,就不需要手動填寫元數據字段。項目如何自動影響其他項目\textbackslash\textbackslash 也屬於這一方面。

\textbackslash section\{4. CHAPTER INTRODUCTION TO THE COMPANY AND
PRODUCT\textbackslash\textbackslash 第四章 公司和產品介紹\}

As one can imagine, one of the unique aspects of this work is its focus
in one specific software solution that tend to be quite flexible in
terms of ease of implementation to different sorts of business. This is
contrary to most use cases regarding PLM implementation where the
business case is the constant and the system is built around it.
Nonetheless, in order to evaluate Odoo as a PLM+MES tool, it is
important to consider an example. The advantage here is that a fictional
company can be picked for this end maximizing the perceived effect of
the software during a simulation. It is considering all those previously
mentioned systems that, for the sake of exemplification, the theoretical
company was organized in the molds of Industry 4.0. This company is a
recently founded small case manufacturing company that uses plastic
injection molding as their primary mean of production and uses additive
manufacturing and fast prototyping as part of their business strategy.
As explained in chapter 2 those are great examples of the path that
industry is taking regarding innovation where mass production is
becoming slowly less important than product variety and time to market.
In order to maximize the tracking of change, most of its business are
based on lower production batches on mainly automated machinery. This
company focus in the production of injected plastic products and rely
heavily in flexible machinery for setting production and prototyping.
Having that in mind, it should be simple enough to simulate continuous
improvement of both product and process to the extent of the evaluated
software. Since this sort of everchanging production is extremely
dependent on information management of all kinds, it must prove to be a
perfect base for applied PLM+MES. In this example the company has
already implemented, since its recent foundation, the Odoo software and
has taken all the necessary training and steps to its proper use. This
allow the removal of the boundaries and limitations that are so common
regarding implementation of the PLM+MES system to an already existing
business, i.e., dependences on legacy systems administrative resistance
to change or integration to old procedures. These are obviously
important, but it is not within the scope of this work. The company aims
to produce a completely new product by the end of the year. After doing
so, the company improved the process of production for said product.
Once there is the need for product improvement, said improvement was
performed as well.

The following diagram (Figure 9) will be taken into consideration as the
path of product development and
improvement:\textbackslash\textbackslash{} This path aims to transmit to
the reader an iterative approach towards development and improvement.
The idea is followed by a product design for which a cycle of
prototyping and redesign takes effect until satisfactory result is
achieved. Then a similar cycle takes place regarding the production
process. At the end of this stage initial development is done and the
actual production can begin. It is at this point that ways of
stablishing the continuous improvement is important. In the case of this
company, we are only considering two main types of upgrade paths, those
being, product upgrade and process upgrade respectively.

如前所述,本研究的一個獨特方面是其專注於一個特定的軟體解決方案,該解決方案在實施不\textbackslash\textbackslash 同類型的業務方面具有相當的靈活性。這與大多數
PLM
實施的使用案\textbackslash\textbackslash 例相反,後者通常是業務案例是常量,系統圍繞其構建。然而,為了評估
Odoo 作為 PLM+MES
工具的重要性,考慮一個範例是很重要的。這裡的優勢在於可以選擇一個虛構公司來達到這一目的\textbackslash\textbackslash,從而在模擬過程中最大化軟體的效果。

考慮到前面提到的所有系統,為了示範,這個理論公司的組織模式符合工業 4.0
的模式。這家公\textbackslash\textbackslash 司是一家新成立的小型製造公司,主要使用塑料注塑成型作為生產手段,並將增材製造和快速原型\textbackslash\textbackslash 製作作為其商業策略的一部分。如第二章所述,這些是行業在創新方面所走路徑的極佳例子,其中\textbackslash\textbackslash 大規模生產的重要性逐漸低於產品多樣性和上市時間。

為了最大限度地追蹤變更,其大部分業務基於低批量生產和主要自動化機器。這家公司專注於\textbackslash\textbackslash 生產注塑塑料產品,並在設定生產和原型製作時嚴重依賴靈活的機器。考慮到這一點,應該足夠簡\textbackslash\textbackslash 單地模擬產品和過程的持續改進,以達到評估軟體的目的。由於這種不斷變化的生產極其依賴各種\textbackslash\textbackslash 資訊管理,因此必須證明其是應用
PLM+MES 的理想基礎。

在這個例子中,該公司自其近期成立以來,已經實施了 Odoo
軟體,並且已經接受了所有必要的\textbackslash\textbackslash 培訓和步驟以正確使用。這樣可以消除已經存在的業務中實施
PLM+MES
系統時常見的邊界和限制\textbackslash\textbackslash,即對遺留系統的依賴、管理上的抗拒變更或與舊程序的整合等。這些顯然很重要,但不在本工作\textbackslash\textbackslash 的範疇內。

該公司的目標是在年底前生產出一種全新的產品。在完成此目標後,該公司改進了該產品的生產\textbackslash\textbackslash 過程。一旦有產品改進的需求,這些改進也隨之進行。
下圖(圖
9)將被考慮為產品開發和改進的路\textbackslash\textbackslash 徑:
\textbackslash begin\{center\} \textbackslash centering
\textbackslash includegraphics{[}width=1\textbackslash linewidth{]}\{image1.png\}
\textbackslash end\{center\}
這條路徑旨在向讀者傳達一種迭代式的開發和改進方法。這個理念隨之而來的是產品設計,對於這一\textbackslash\textbackslash 設計,進行原型製作和重新設計的循環,直到達到滿意的結果。然後,類似的循環發生在生產過程中。\textbackslash\textbackslash 在這一階段結束時,初步開發完成,實際生產可以開始。

在此時,確立持續改進的方法是很重要的。對於這家公司,我們只考慮兩種主要的升級路徑,分\textbackslash\textbackslash 別是產品升級和過程升級。

4.1. The products and processe Change and effect are the focus of the
PLM+MES implementation as such the subject of said change would ideally
be something that could afford a reasonable amount of freedom of design.
Although the effects of a well implemented PLM+MES should be substantial
even in rigid manufacturing environments, where the change is extremely
limited, the system will produce much more perceivable change in an
enterprise that thrives in innovation because there will be more
opportunities to improve the system and gain feedback. From the
perspective of improvement, if you compare a product that is a result
from sheet metal stamping (Figure 10) to an equivalent product that is
the result of a CNC milling procedure (Figure 11) it is easy to perceive
that the CNC milled product is more welcomingto upgrades. While the
stamping is low cost (by comparison) it depends on heavy high precision
metal dyes that are extremely expensive to produce. This means that the
cost of enacting change to it is much higher and thus the effect of a
system that thrives on tracking change becomes limited.
s\textbackslash\textbackslash4.1.
產品和流程\textbackslash\textbackslash{} 變更及其效果是 PLM+MES
實施的重點,因此,理想情況下,這些變更的對象應該是設計上具有相當自由度的東西。儘管在變\textbackslash\textbackslash 更極其有限的嚴格製造環境中,良好實施的
PLM+MES
也應產生顯著效果,但在依賴創新發展的\textbackslash\textbackslash 企業中,系統會產生更多可感知的變化,因為會有更多改進系統和獲得反饋的機會。

從改進的角度來看,如果比較由金屬板衝壓(圖 10)生成的產品與由 CNC
銑削工序(圖
11)\textbackslash\textbackslash 生成的等效產品,可以很容易地看出,CNC
銑削的產品更容易接受升級。雖然衝壓的成本相對較低\textbackslash\textbackslash,但它依賴於製造成本極高的高精度金屬模具。這意味著對其進行變更的成本要高得多,因此,一\textbackslash\textbackslash 個依賴追蹤變更的系統的效果會受到限制。
\textbackslash begin\{center\} \textbackslash centering
\textbackslash includegraphics{[}width=1\textbackslash linewidth{]}\{3.png\}
\textbackslash end\{center\}

In the case of this fictional company, it has been determined that the
best way to exemplify the PLM+MES effects would be to have products
designed around plastic injection molding. It might seem unintuitive at
first to consider this manufacturing procedure, like the stamping
procedure previously described, since it too depends on high precision
molds during production. However, the main differences between the two
is regarding ease of prototyping and the cost of upgrading. Injection
molding is a broad and complex field of engineering that involves a huge
variety of materials and methods, little of which is of the concern of
this work. It is however relevant to point out that for the most part,
the pressures involved in the injection molding are one order of
magnitude lower than the when we are dealing with steel; softer
materials can beused on their molds like CNC milled aluminum. At the
same time, new advancements in the field of additive manufacturing have
made possible to prototype plastic parts with much closer physical
characteristics to the end result of a injected piece. Sometimes even
prototype molds (Figure 12) can be used for a lower volume test runs
during process upgrades. 對於這家虛構公司來說,已確定最好的方式來示範
PLM+MES
的效果是設計圍繞塑料注塑成型的\textbackslash\textbackslash 產品。最初考慮這種製造工藝可能會顯得不太直觀,就像之前描述的衝壓工藝一樣,因為它也依賴\textbackslash\textbackslash 於生產過程中的高精度模具。然而,兩者之間的主要區別在於原型製作的便利性和升級的成本。

注塑成型是一個廣泛且複雜的工程領域,涉及各種各樣的材料和方法,其中很少部分是本研究\textbackslash\textbackslash 關注的。然而,值得指出的是,大多數情況下,注塑成型所涉及的壓力比處理鋼材時低一個數量級;\textbackslash\textbackslash 因此可以在模具上使用較軟的材料,如
CNC
銑削鋁。同時,增材製造領域的新進展使得原型製作\textbackslash\textbackslash 的塑料零件在物理特性上與注塑件的最終結果更為接近。有時甚至可以使用原型模具(圖
12)進行\textbackslash\textbackslash 低容量的測試運行,以在過程升級期間進行測試。

\textbackslash begin\{center\} \textbackslash centering
\textbackslash includegraphics{[}width=1\textbackslash linewidth{]}\{4.png\}
\textbackslash end\{center\} Additive manufacturing has become an
incredible tool for ultra-flexible production. This mindset of
continuous improvement, especially when regarding prototyping and
iterative design, is a hallmark of the lean mentality that is so
relevant in the modern industry. As mentioned in the previous section,
in this case study it is considered the creation of a new product and
its production process by the fictional company. This product consists
in a plastic small form factor computer case, composed of 3 different
parts (Figure 13) that are expected to be designed and prototyped
considering combination of additive manufacturing and CNC milling
towards a plastic injection molding production.

增材製造已成為超靈活生產的一個不可思議的工具。這種持續改進的思維方式,特別是在原型\textbackslash\textbackslash 製作和迭代設計方面,是現代工業中非常重要的精益思維的特徵。

如前一節所述,在這個案例研究中,虛構公司考慮創造一\textbackslash\textbackslash 個新產品及其生產過程。該產品是一個由三個不同部件組成的小型塑料電腦機殼(圖
13),預計\textbackslash\textbackslash 會考慮結合增材製造和 CNC
銑削進行設計和原型製作,最終實現塑料注塑成型生產。

\textbackslash begin\{center\} \textbackslash centering
\textbackslash includegraphics{[}width=1\textbackslash linewidth{]}\{5.png\}
\textbackslash end\{center\} 4.1.1. Part A PART-A (Figure 14) is the
core structure of the computer case. It is expected to comport all the
pieces necessary for the proper function of the small form factor
computer in question. To this end a raw material A was selected to be
Acrylonitrile Butadiene Styrene (ABS) this is an opaque thermoplastic
polymer and an engineering grade plastic. It is commonly used to produce
electronic parts such as phone adaptors, keyboard keys and wall socket
plastic guards.The main reasons for choosing this material specifically
are its toughness, its good dimensional stability (resistance to change
dimensions after cooling), its high impact resistance and surface
hardness. Finally, it is also commonly available in the form of 3D
printing filament for extrusion 3D printers which should prove to be
quite useful during prototyping. \textbackslash\textbackslash4.1.1.
部件A\textbackslash\textbackslash{}
部件A(圖14)是電腦機殼的核心結構。預計它將包含所有必要的部\textbackslash\textbackslash 件,以確保這個小型電腦的正常運行。為此,選擇了原材料A,即丙烯腈-丁二烯-苯乙烯(ABS),\textbackslash\textbackslash 這是一種不透明的熱塑性聚合物和工程級塑料。它通常用於生產電子零件,如手機適配器、鍵盤鍵\textbackslash\textbackslash 帽和插座塑料護罩。

具體選擇這種材料的主要原因是其堅韌性、\textbackslash\textbackslash 良好的尺寸穩定性(冷卻後的尺寸變化抵抗力)、高抗衝擊性和表面硬度。最後,它也常見於3D\textbackslash\textbackslash 打印機的擠出3D打印絲材,這在原型製作過程中應該非常有用。
\textbackslash begin\{center\} \textbackslash centering
\textbackslash includegraphics{[}width=1\textbackslash linewidth{]}\{6.png\}
\textbackslash end\{center\} 4.1.2. Parts B and C Parts B and C are lids
that should snap into place, closing the system. These are very simple
pieces and require a certain level of elasticity so it can deform to
assure a screwless assembly. These two identical parts are going to be
made with Thermoplastic Polyurethane (TPU), because of its elastic
nature and great tensile and tear strength. This sort of polymer is
often used to produce parts that demand a rubber-like elasticity. TPU
performs well at high temperatures and is commonly used in power tools,
cable insulations and sporting goods. Finally, TPU is also available in
the form of filament for 3D printers which, for the simulation, will be
used for prototyping. \textbackslash\textbackslash4.1.2.
部件B和C\textbackslash\textbackslash{}
部件B和C是應該卡入到位的蓋子,負責封閉系統。這些都是非常簡單的\textbackslash\textbackslash 部件,需要一定程度的彈性,以便在無需螺絲的情況下裝配。這兩個相同的部件將使用熱塑性聚氨\textbackslash\textbackslash 酯(TPU)製作,因為其彈性特性以及優秀的拉伸和撕裂強度。這種聚合物常用於需要橡膠般彈性\textbackslash\textbackslash 的部件製作。TPU在高溫下表現良好,常用於電動工具、電纜絕緣和運動用品中。最後,TPU也以\textbackslash\textbackslash3D打印絲材的形式存在,這在模擬中將用於原型製作。
\textbackslash begin\{center\} \textbackslash centering
\textbackslash includegraphics{[}width=1\textbackslash linewidth{]}\{7.png\}
\textbackslash end\{center\} 4.1.3. Molds Ideally all molds should be
made of steel, for longevity of the mold and product quality. That being
said, the injected plastics that are being selected for all parts are
not so pressure dependent and their forms are not so complex, so it is
assumed that aluminum molds made with a precision CNC machining should
suffice to produce said parts. It is also assumed that all molds are
simple enough to be prototyped using 3D printing. Although this is not
always true, it was determined representative enough for this
simulation. The type of material used in those prototypes is high
temperature resign cured using an SLA 3DPrinter. Additionally, the mold
will be considered the main physical aspect to be developed when
regarding the production process because it something that directly
affects the production as well as something that can be produced in
house and tracked as a product would. \textbackslash\textbackslash4.1.3.
模具\textbackslash\textbackslash{}
理想情況下,所有模具都應該由鋼製成,以保證模具的壽命和產品\textbackslash\textbackslash 質量。儘管如此,由於選用的注塑塑料對壓力不那麼敏感,且其形狀不那麼複雜,因此假設使用\textbackslash\textbackslash 精密
CNC 加工製作的鋁模具應該足以生產所需部件。

同樣假設所有模具簡單到可以用 3D
打印進行原型製作。儘管這並非總是如此,但這種假設對\textbackslash\textbackslash 於此次模擬足夠具代表性。這些原型中使用的材料是高溫樹脂,通過
SLA 3D
打印機進行固化。\textbackslash\textbackslash 此外,模具將被視為主要的物理開發對象,因為它直接影響生產過程,可以在公司內部生產並作\textbackslash\textbackslash 為產品進行跟蹤。
4.2. What is analized during the simulation Taking into consideration
the diagram, shown in Figure 9, as well as the main aspects of a
successful integration of PLM and MES as described in the section 3.1,
this experiment aims to produce commentary regarding the following
relevant questions in Table 1. \textbackslash\textbackslash4.2.
模擬過程中的分析內容\textbackslash\textbackslash{} 考慮到圖 9
中所示的圖表以及第 3.1 節中描述的成功集成 PLM 和 MES
的主要方面,本實驗旨在對\textbackslash\textbackslash 表 1
中的以下相關問題進行評論。 \textbackslash begin\{center\}
\textbackslash centering
\textbackslash includegraphics{[}width=1\textbackslash linewidth{]}\{8.png\}
\textbackslash end\{center\} \textbackslash section\{5.CHAPTER THE ODOO
SOFTWARE\textbackslash\textbackslash 第5章 Odoo 軟體\} 5.1. Introduction
to the Odoo software Odoo is a commercial business management software
with strong ties to the open source community. Initially started as open
source ERP software becoming well received as an affordable and
intuitive package that thrived on integration and expandability. Since
then, as the company experienced accelerated growth, it shifted their
business model to include an enterprise paid version as well as an
online service. As mentioned in the section 2.2, modern ERP systems are
usually modular and, in the case of Odoo, this modularity is
particularly evident due to the incredible amount of expansion provided
by community developed modules as well as company developed modules that
are highly integrated. This extendibility is what makes this software so
relevant to the topic of PLM+MES integration since there are present
modules for PLM as well as noticeable MES functionalities within their
manufacturing modules. Within the scope of this thesis, the objective is
to utilize this software on the management of the previously mentioned
fictional company and draw conclusions regarding how effective the
integration of PLM and MES is already present within this system. 5.1.
Odoo 軟體介紹 Odoo 是一款與開源社區關係密切的商業管理軟體。最初作為開源
ERP
\textbackslash\textbackslash 軟體推出,因其經濟實惠且直觀的特性而廣受好評,並在整合性和可擴展性方面表現出色。隨著\textbackslash\textbackslash 公司的快速增長,他們的商業模式轉變為包括企業付費版本以及線上服務。

如第2.2節所述,現代ERP系統通常是模組化的,而在Odoo的情況下,這種模組化特性特別明\textbackslash\textbackslash 顯,這歸功於社區開發模組和公司開發模組提供的豐富擴展性。這種可擴展性使得該軟體在PLM\textbackslash\textbackslash+MES整合主題中顯得尤為重要,因為它包含了PLM模組以及其製造模組中的顯著ME\textbackslash\textbackslash S功能。

在本論文的範圍內,目標是利用這款軟體來管理前述的虛構公司,並對該系統內部PLM和ME\textbackslash\textbackslash S整合的有效性進行結論性分析。
5.1.1. How it works The software can be installed in most x86 computers
and it supports several operating systems including windows and all the
main Linux distributions. Ideally, the Odoo software is installed in a
computer connected to a local area network and starts a SQL database
that holds all the necessary information and files produced by the
business (Figure 16). Said computer works essentially as a server and
accessed via a browser by the other machines present in the network.
This computer can be a dedicated server or a working desktop in use, but
it is important to remember that it must remain ON and connected
throughout the entire time the software is required to function. 5.1.1.
工作原理 該軟體可以安裝在大多數 x86 電腦上,並支持多個操作系統,包括
Windows 和所有主要的 Linux 發行版。

理想情況下,Odoo 軟體安裝在連接到本地區域網絡的電腦上,並啟動一個 SQL
資料庫,該資\textbackslash\textbackslash 料庫保存企業產生的所有必要信息和文件(圖
16)。這台電腦本質上充當伺服器,其他網絡中的\textbackslash\textbackslash 機器通過瀏覽器訪問它。這台電腦可以是專用伺服器或在使用中的工作桌面,但重要的是要記住,\textbackslash\textbackslash 它必須在軟體運行期間保持開機和連接。

\textbackslash end\{document\}

工單是製造操作員與Odoo交互的主要形式,它呈現操作項指定的所有指令,以及對其完成的控制。當
WO 發生時,操作員通過介面發出信號,發出信號,發出信號,完成所有 WO
后,可
